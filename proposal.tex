
\documentclass[11pt,a4paper]{article}
\usepackage[utf8]{inputenc}
\usepackage[T1]{fontenc}

\usepackage[dvipsnames,usenames]{color}
\usepackage[colorlinks=true,urlcolor=Blue,citecolor=Green,linkcolor=BrickRed]{hyperref}
\usepackage[usenames,dvipsnames]{xcolor}
\urlstyle{same}

\usepackage{amsthm,amsmath,amssymb,mathabx}
\usepackage{fullpage}
\usepackage[ruled,noend,linesnumbered]{algorithm2e}     
\usepackage{enumerate,comment}
\usepackage{url}
\usepackage[capitalise]{cleveref}
\usepackage{todonotes}
\usepackage{tikz}
\usepackage[noadjust]{cite}
\usepackage{xspace}
\usepackage{graphicx}
\usepackage{float}
\usepackage{amsfonts}
\usepackage{color}
\usepackage{array}

%\pagestyle{plain}

\newcommand{\ceil}[1]{\left\lceil #1 \right\rceil}
\newcommand{\floor}[1]{\left\lfloor #1 \right\rfloor}
\newcommand{\poly}{\text{poly}}

\newcommand{\Oh}{{O}}

% For cleverref compatibility
\newtheorem{theorem}{Theorem}[section]
\newtheorem{algo}{Algorithm}[section]
\newtheorem{corollary}{Corollary}
\newtheorem{lemma}{Lemma}
\newtheorem{observation}{Observation}
\newtheorem{fact}[theorem]{Fact}
\theoremstyle{definition}   
\newtheorem{definition}{Definition}
\usepackage{authblk}
\theoremstyle{remark}
\newtheorem{example}[theorem]{Example}
\newtheorem*{claim}{Claim}
\newtheorem{case}{Case}
\newtheorem{step}{Step}[section]

\newtheorem{property}[theorem]{Property}
\newcolumntype{R}[1]{>{\raggedleft\let\newline\\\arraybackslash\hspace{0pt}}m{#1}}

\begin{document}

\title{Dispersion on Trees}
\author{An M.Sc. Proposal by  Nadav Krasnopolsky}


\date{}
\maketitle

\begin{abstract}
In the dispersion problem we want to select $k$ nodes of a given graph as to maximize
the minimum distance between any two chosen nodes. This can be seen as a generalization
of the independent set problem, where the goal is to select nodes so that the minimum distance
is larger than 1.
We design an optimal $\Oh(n)$ time algorithm for the dispersion problem on trees consisting
of $n$ nodes, thus improving the previous $\Oh(n\log n)$ time solution. 

Then we consider the weighted version. 
For the decision
version, where the goal is to find a set of nodes of
total weight at least $k$ such that any two nodes are at distance at least $\lambda$, we present tight $\Theta(n\log n)$ upper and lower bounds. For the optimization version, where we wish to maximize  $\lambda$, we present an
 $\Oh(n\log^2n)$ upper bound improving the previous $\Oh(n\log^4 n)$ time solution. 
\end{abstract}

\section{Introduction}
Facility location is a family of problems, where the goal is to place
a number of facilities as to minimize the total cost while preserving
given constraints. In its basic version, called the metric $k$-center
problem, we wish to designate up to $k$ nodes of a given weighted graph
to be facilities, as to minimize the maximum distance of a node to its
closes facility. Even this simplest version is already NP-hard \cite{Vazirani2003}. However, a simple greedy 2-approximation algorithm is known \cite{Gonzalez1985}. The situation for different
objective functions or more restrictions on the placement is of course
more complex, but nevertheless one can often design an efficient
approximation algorithm, see \cite{DavidB.Shmoys1997} and  references
therein.

Given that facility location problems are usually if not always
NP-hard, it makes sense to consider them on restricted 
graphs families. The simplest yet still non-trivial such a family are trees on
$n$ nodes. There are multiple possible versions of the $k$-center
problem on trees: the facilities can be either only the nodes of the
tree or any points on an edge, and we might either minimize the
distance of any node or any point on an edge to its closest facility.
All these versions have been extensively studied, culminating in an
$\Oh(n)$ time and space algorithm given by Frederickson \cite{Frederickson1991a} that
solves all but one of them. This was an improvement on the previous
$\Oh(n\log n)$ time algorithm by Frederickson and Johnson \cite{Frederickson1983}, which in turn was an improvement
on the $\Oh(n\log^2n)$ time algorithm of Megiddo et al. \cite{Megiddo1981}. 

The weighted version
of these problems has also been studied. In this version, every node has an associated weight and the
distance is multiplied by the corresponding weight. The currently fastest solution is the $\Oh(n\log^2n)$ time algorithm of Megiddo and Tamir %cite Cole and
\cite{Megiddo1983}. Among the facility location problems on trees there
is the max-min tree $k$-partitioning \cite{Perl1981} and the min-max tree
$k$-partitioning \cite{Becker1982}, both solved by Frederickson in $\Oh(n)$ time
\cite{Frederickson1991} using a clever approach based on the parametric search that
he has later extended to solve the $k$-center problem.

In this paper we focus on the max-min tree $k$-dispersion problem that falls within the class of facility
location problems on trees. In this problem, the goal
is to select $k$ nodes so as to maximize the minimum distance between any
two chosen nodes. In other words, we wish to select $k$ nodes that are as spread-apart as possible. 
This generalizes the maximum independent set
problem by binary searching for the largest value of $k$ for which the
minimum distance is at least 2. The best previously known algorithm
for the max-min tree $k$-dispersion was given by Bhattacharya and Houle
\cite{Bhattacharya1991} and takes $\Oh(n\log n)$ time, improving on an earlier $\Oh(kn+n\log n)$
solution for the one dimensional case given by Wang and Kuo~\cite{Wang1988} . Using the
framework of Frederickson, we construct an optimal $\Oh(n)$ time algorithm
for this problem. In our solution we also develop a slightly modified
version of this framework, which might be simpler to understand. Then
we move the weighted version, where the goal is to select nodes with
 total weight  at last $k$. There, the best previously known
solution requires $\Oh(n\log^4 n)$ time \cite{Bhattacharya1991}. We present a significantly
faster $\Oh(n\log^2 n)$ time algorithm. We also show that the decision
version, where the goal is to check if there exists a set of of nodes
total weight $k$ such that any two nodes are at distance at least $b$,
requires $\Omega(n\log n)$ time in the algebraic decision tree model. We
also present a matching $\Oh(n\log n)$ time algorithm, which is then used
to construct our $\Oh(n\log^2n)$ time solution for the weighted max-min
tree $k$-dispersion using the standard parametric search paradigm of Fredrickson. 

\bibliographystyle{alpha}
\bibliography{dispersion}



\end{document}