hjfxhbvkj\documentclass[11pt]{article}
\usepackage[utf8]{inputenc}
\usepackage[T1]{fontenc}

\usepackage{amsthm,amsmath,amssymb, mathabx}
\usepackage{fullpage}
\usepackage[ruled,noend,linesnumbered]{algorithm2e}     
\usepackage{enumerate,comment}
\usepackage{url}
\usepackage[capitalise]{cleveref}
\usepackage{todonotes}
\usepackage{tikz}
\usepackage[noadjust]{cite}

\usepackage{graphicx, subfigure}
\usepackage{array}
\usepackage{xspace}
\usepackage{graphicx}
\usepackage{float}
\usepackage{amsfonts}

\usepackage{color}

%\usepackage{algorithm}      % http://ctan.org/pkg/algorithms
%\usepackage{algpseudocode}  % http://ctan.org/pkg/algorithmicx
\usetikzlibrary{trees}

%\pagestyle{plain}

\newcommand{\ceil}[1]{\left\lceil #1 \right\rceil}
\newcommand{\floor}[1]{\left\lfloor #1 \right\rfloor}
\newcommand{\poly}{\text{poly}}

\newcommand{\Oh}{{O}}
\newcommand{\common}{\mathsf{common}}
\newcommand{\encode}{\mathcal{L}}
\newcommand{\position}{\mathsf{position}}
\newcommand{\leaves}{\mathsf{leaves}}
\newcommand{\diff}{\mathsf{diff}}
\newcommand{\id}{\mathsf{id}}
\newcommand{\lightdepth}{\mathsf{lightdepth}}
\newcommand{{\appx}}[2][2]{\lfloor{#2}\rfloor _{#1}}
\newcommand{\pre}{\mathsf{pre}}
\newcommand{\lightrange}[1]{\mathsf{L}_{#1}}
\newcommand{\heavy}{\mathsf{heavy}}
\newcommand{\lightsize}{\mathsf{ligthsize}}
\newcommand{\apex}{\mathsf{apex}}
\newcommand{\size}{\mathsf{size}}
\newcommand{\NCA}{\mathsf{NCA}}
\newcommand{\nca}{\text{NCA}}
\newcommand{\NCSA}{\mathsf{NCSA}}
\newcommand{\NCH}{\mathsf{NCH}}
\newcommand{\pow}{\mathsf{pow}}
\newcommand{\distance}{\mathsf{d}}
\newcommand{\nil}{\mathsf{nil}}
\newcommand{\half}{\frac{1}{2}}
\newcommand{\children}{\mathsf{children}}
\newcommand{\parent}{\mathsf{parent}}

\newcommand{\dist}{\text{dist}}
\newcommand{\tlab}{L}
\newcommand{\tlabmax}{M}
\newcommand{\tlabbound}{\mathcal{M}}
\newcommand{\scheme}{A}
\newcommand{\allschemes}{\mathcal{A}}
\newcommand{\lightcount}{\lightdepth}
\newcommand{\ncadist}{\mathsf{NCA}\text{-}\mathsf{distance}}
\newcommand{\treeroot}{\mathsf{root}}
\newcommand{\collapsed}{\mathcal{C}}
\newcommand{\hphead}{\mathsf{head}}



% For cleverref compatibility
\newtheorem{theorem}{Theorem}[section]
\newtheorem{algo}{Algorithm}[section]
\newtheorem{corollary}{Corollary}
\newtheorem{lemma}[theorem]{Lemma}
\newtheorem{observation}[theorem]{Observation}
\newtheorem{fact}[theorem]{Fact}
\theoremstyle{definition}   
\newtheorem{definition}[theorem]{Definition}
\usepackage{authblk}
\theoremstyle{remark}
\newtheorem{example}[theorem]{Example}
\newtheorem*{claim}{Claim}
\newtheorem{case}{Case}

\newtheorem{property}[theorem]{Property}
\newcolumntype{R}[1]{>{\raggedleft\let\newline\\\arraybackslash\hspace{0pt}}m{#1}}

\begin{document}

\title{Dispersion on Trees}
\author[1]{}
\affil[1]{University of Haifa, Israel}


\date{}
\maketitle

\begin{abstract}
bla bla 
\end{abstract}

\section{Introduction}
\begin{definition}
\emph{Max-Min Dispersion Problem (MMDP)} Given a tree $T$ with non-negative edge lengths, and a natural number $p$, find a subset $P\subseteq V$ of size $p$ s.t $f(P)=min{\scriptscriptstyle \forall u,v\in P,u\neq v}d(u,v)$ is maximized.
\end{definition}
\begin{definition}
\emph{MMDP Feasibility Test} Given a MMDP instance and a number $\lambda\geq0$ decide if there exists a subset $P\subseteq V$ of size $p$ s.t $f(P)\geq\lambda$.
\end{definition}
\begin{definition}
\emph{Weighted Max-Min Dispersion Problem (WMMDP)} Given a tree $T$ with non-negative vertex weights, non-negative edge lengths, and a real number $w\geq0$, find a subset $P$ of the vertices of $T$ s.t $W(P)\geq w$ (where $W(P)$ is the sum of the weights of the vertices of $P$), and $f(P)$ is maximized.
\end{definition}
\begin{definition}
\emph{WMMDP Feasibility Test} Given a WMMDP instance and a number $\lambda\geq0$ decide if there exists a subset $P\subseteq V$ s.t $W(P)\geq w$ and $f(P)\geq\lambda$.
\end{definition}


\section{The unweighted case}
We first describe how to solve the feasibility test in linear time, and then use this algorithm to solve the MMDP.
\subsection{Linear algorithm for the MMDP Feasibility Test}
%\begin{algo} \label{unWeightedFeasibilityAlgo}
We show a recursive linear algorithm.
At each step of the recursion, we would like to provide the maximal valid solution for the current subtree, given some $\lambda$. I.e, we would like to return $P$, a maximal subset of the vertices of the subtree, s.t   $P\subseteq V$ of size $p$ s.t $f(P)\geq\lambda$.
We are given a root vertex $v$ and its children nodes $v_{1},v_{2},...,v_{k}$, and for each child we are given a valid solution for the MMDP feasibility test on its subtree. We would like to produce a valid solution for the feasibility test on $v\text{'s subtree}$.
Denote by $P_{1},...,P_{k}$ the solutions for the feasibility test on each subtree rooted at a child of $v$.
For any subtree of $T$ rooted at node $r$, and a valid solution $P$ for the MMDP feasibility test on the subtree, we call a node $u\in P$, s.t $d(r,u)\leq\frac{\lambda}{2}$, a \textcolor{blue}{blue} node of the subtree. We call the vertex in $P$ that is closest to $r$, but isn't blue, \textcolor{green}{green}.
Note that each $P_{i}$ contains at most one blue vertex and one green vertex.

\paragraph{The recursion step} Given $P_{1},...,P_{k}$, we would like to produce a solution for $v$'s subtree.
\begin{enumerate}
\item Put in $P$ all the vertices in $P_{1},...,P_{k}$, except for the blue vertices.
\item Take all blue nodes $u$ s.t $d(u,v)> \frac{\lambda}{2}$
\item Take $u'$, the blue node farthest from $v$ s.t $d(u',v)\leq \frac{\lambda}{2}$, if it exists, and if $d(u',x)\leq \lambda$, where $x$ is the closest node to the root $v$ we have chosen so far.
\item Decide weather to take $v$ (the root) to the solution by looking at the closest vertex to it we have already put in $P$.
\end{enumerate}

We start by putting in $P$ all the vertices in $P_{1},...,P_{k}$, except for the blue vertices.
Now we need to decide which blue nodes we should put in $P$ and if we take $v$. We would like to take as many blue nodes as we can, and still maintain a dispersion of $\lambda$.
Take all blue nodes $u$ s.t $d(u,v)> \frac{\lambda}{2}$, and also take $u'$, the blue node farthest from $v$ that does not maintain this, i.e $d(u',v)\leq \frac{\lambda}{2}$, if it exists. We can decide if we take $v$ to the solution in constant time by looking at the closest vertex to it we have already put in $P$.
%\end{algo}

\begin{definition}
\emph{Active and Inactive nodes} We will call a vertex $v$ of the tree \emph{inactive} if there is a vertex $u$ we already chose to be in $P$, s.t $d(u,v)<\lambda$. We will call $v$ \emph{active} if there is no such vertex $u$.
\end{definition}
\begin{lemma} \label{greenNodesLemma}
For any subtree of $T$ rooted at node $r$, let $u$ be the most distant active vertex from $r$ in the subtree. if $d(u,r)\ge\frac{\lambda}{2}$, then there exists a solution to the MMDP, for which $f(P)\leq\lambda$ and $u$ is in the solution.
\end{lemma}
\begin{proof}
Assume for contradiction that there is no solution $P$ for which $f(P)\leq\lambda$ s.t $u\in P$. We consider the stage of a Denote the closest vertex to $u$ in $P$ by $u'$. We assume that we cannot replace $u'$ with $u$ and still get a valid solution. This means that there is a vertex $x$, s.t $x\in P$ and $d(u,x)<\lambda$. 
We know that:
\corollary{\label{corol1} $d(r,u)\ge\frac{\lambda}{2}$ (by definition)}
\corollary{\label{corol2} $d(u,u')<\lambda$ (because otherwise we could definitely add $u$ to the solution as $u'$ is defined as closest to $u$ in $P$)}
\corollary{\label{corol3} $d(x,u')\ge\lambda$ (since $x$ and $u'$ are both in $P$)}
\corollary{\label{corol4} $d(x,u)<\lambda$ (by definition) $d(r,u)>d(r,u')$ (since otherwise $u$ is not the active vertex most distant from $r$).}
Let us look at the possible cases:
\begin{case} Both $u'$ and $x$ are in the subtree rooted at $r$:
$d(u,u')=d(u,r)+d(r,u')<\lambda$ (due to corollary \ref{corol2}) $\Rightarrow d(r,u')<\frac{\lambda}{2}$ (due to corollary \ref{corol1}, and to the fact that by definition $u$ is closer to $r$ than $u'$), and similarly $d(r,x)<\frac{\lambda}{2} \Rightarrow d(x,u')<\lambda$ in contradiction to corollary \ref{corol3}. 
\end{case}
\begin{case} $u'$ is in the subtree, and $x$ is not:
This means that the paths from $u$ and $u'$ to $x$ go through $r$. We have that $d(r,u)>d(r,u')$ (since otherwise $u$ is not the active vertex most distant from $r$), which implies that $d(r,u)+d(r,x)>d(r,u')+d(r,x)\Rightarrow d(u,x)>d(u',x)$ in contradiction to corollary \ref{corol3} and corollary \ref{corol4}.
\end{case}
\begin{case} $x$ is in the subtree and $u'$ is not:
We have that $d(r,u)>d(r,x)$ (since otherwise $u$ is not the active vertex most distant from $r$), and corollary \ref{corol2} implies that $d(u',r)+d(r,u)<\lambda$, so we get that $d(u',r)+d(r,x)<\lambda$ in contradiction to corollary \ref{corol3}.
\end{case}
\begin{case} Both $x$ and $u'$ are not in the subtree:
\begin{case} $x$ is on the path from $u$ to $u'$:
We have that $d(u,u')=d(u,r)+d(r,x)+d(x,u')<\lambda\,\,\Rightarrow\,\,d(x,u')<\lambda$ in contradiction to corollary \ref{corol3}.
\end{case}
\begin{case} $u'$ is on the path from $u$ to $x$:
We have that $d(u,x)=d(u,r)+d(r,u')+d(u',x)<\lambda\Rightarrow d(x,u')<\lambda$ in contradiction to corollary \ref{corol3}.
\end{case}
\begin{case} neither $u'$ is on the path from $u$ to $x$ nor $x$ is on the path from $u$ to $u'$:
$d(u,x)=d(x,r)+d(r,u)<\lambda\Rightarrow(\text{by 1) }d(x,r)\le\frac{\lambda}{2}$ and similarly we get that $d(u',r)\le\frac{\lambda}{2}\,\,\,\Rightarrow\,\,\,d(u',x)\le\lambda$. 
\end{case}
\end{case}
\end{proof}
Lemma \ref{greenNodesLemma} implies the correctness of the algorithm.



\bibliographystyle{plain}
\bibliography{dispersion}



\end{document}